\section{16-10-2014}
\subsection{Markov Chains 1.8}

\begin{prop}
Give an example where the limit \(p_{ij}^{(n)}\) fails to converge.
\end{prop}
\begin{proof}
Consider the two-state chain with transition matrix
\begin{gather*}
P=\begin{pmatrix}0&1\\ 1&0\end{pmatrix}.
\end{gather*}

Then \begin{gather*}
P^2=\begin{pmatrix}0&1\\ 1&0\end{pmatrix}\begin{pmatrix}0&1\\ 1&0\end{pmatrix}=I.
\end{gather*}
So \(P^2=I\), so \(P^{2n}=I.\) And \(P^{2n+1}=P.\) Thus \(p_{ij}^{(n)}\) fails to converge for all \(i,j\).
\end{proof}

\begin{defn}
We call a state \(i\) \emph{aperiodic} if \(p_{ii}^{(n)}>0\) for all sufficiently large \(n\).
\end{defn}

\begin{thm}
A state \(i\) is aperiodic if and only if the set
\(\{n\geq 0: p_{ii}^{(n)}>0\}\) has no common divisor other than \(1.\)
\end{thm}

\begin{thm}[Lemma 1.8.2]
Suppose \(P\) is irreducible and has an aperiodic state \(i\). Then, for all states \(j\) and \(k\), \(p_{jk}^{(n)}>0\) for all sufficiently large \(n\). In particular, all states are aperiodic.
\end{thm}

\begin{thm}[Theorem 1.8.3]
Let \(P\) be irreducible and aperiodic, and suppose that \(P\) has an invariant distribution \(\pi \). Let \(\lambda \) be any distribution. Suppose that
\((X_{n})_{n\geq 0}\) is Markov\((\lambda ,P)\). Then for all \(j\)
\[
\Bbb{P}(X_{n}=j)\rightarrow \pi _{j} \qquad n\rightarrow \infty .
\]
In particular, for all \(i,j\)
\[
p_{ij}^{(n)}\rightarrow \pi _{j} \qquad n\rightarrow \infty 
\]
\end{thm}

\subsection{Representation Theory Reader Week 39}

Let \(k\) be a field. And let \(A\) be a \(k\)-algebra.

\begin{defn}
An element \(a\in A\) is called an \emph{idempotent element} or an \emph{idempotent} if \(a^2=a.\)
\end{defn}

\begin{prop}
We have that \(0\) and \(1\) are idempotents of \(A\).
\end{prop}

\begin{defn}
Two idempotents \(a,b\in A\) are called \emph{orthogonal} if \(ab=0\) and \(ba=0\).
\end{defn}

\begin{prop}
If \(a,b\) are orthogonal, then \(a+b\) is an idempotent.
\end{prop}

\begin{proof}
Showing that \(a+b\) is idempotent, is equivalent with showing that
\[
(a+b)^2=a+b
\]
which is equivalent with showing that
\[
a+ab+ba+b=a+b
\]
which in turn is equivalent with showing that
\[
ab+ba=0.
\]
It suffices to show that
\[
ab=0=ba.
\]
By hypothesis we have that $a,b$ are orthogonal, and therefore our last statement follows by definition.
\end{proof}

\begin{defn}
A set of mutually orthogonal idempotents \(a_{1},\ldots ,a_{n}\in A\) is called \emph{complete} if
\[
1=a_{1}+\cdots +a_{n}.
\]
\end{defn}

\begin{defn}
An idempotent \(a\in Z(A)\) is called a \emph{central idempotent} of \(A\).
\end{defn}

\begin{defn}
A nonzero idempotent \(a\in A\) is called \emph{minimal} if any decomposition of \(a\) as a sum of two orthogonal idempotents \(a=b+c\) implies that \(b=0\) or \(c=0\).
\end{defn}

\begin{prop}
If \(V\) is a \(k\)-vector space, then \(\End_{k}(V)\) is a \(k\)-algebra.
\end{prop}

\begin{proof}
Showing that \(\End_{k}(V)\) is a \(k\)-algebra is equivalent with showing that there exists a ring homomorphism
\[
f: k \rightarrow  Z(\End_{k}(V))
\]
which is equivalent with showing that there exists a map \(f\) such that

\begin{enumerate}
  \item \(f(\alpha )L=Lf(\alpha ) \qquad \forall \alpha \in k,\forall L\in \End_{k}(V)\)
  \item \(f(\alpha )f(\beta )=f(\alpha \beta ) \qquad \forall \alpha ,\beta \in k\)
  \item \(f(\alpha )+f(\beta )=f(\alpha +\beta ) \qquad \forall \alpha ,\beta \in k\)
\end{enumerate}

Remember that the condition in 1) implies that \(f(\alpha )\) must be a scalar. Let's try the mapping \(f(\alpha )=\alpha I\) and rewrite our conditions:

\begin{enumerate}
  \item \(\alpha I\circ L=L\circ \alpha I \qquad \forall \alpha \in k,\forall L\in \End_{k}(V)\)
  \item \(\alpha I\beta I=(\alpha \beta )I \qquad \forall \alpha ,\beta \in k\)
  \item \((\alpha +\beta )I=\alpha I+\beta I \qquad \forall \alpha ,\beta \in k\)
\end{enumerate}

All those conditions hold directly.
\end{proof}

\begin{prop}
If \(V\) is a representation of a \(k\)-algebra \(A\), then \(\End_A(V)\) is a \(k\)-algebra.
\end{prop}

\begin{proof}
Showing that \(\End_A(V)\) is a \(k\)-algebra is equivalent with showing that there exists a ring homomorphism
\[
f: k \rightarrow  Z(\End_A(V))
\]
which is equivalent with showing that there exists a map \(f\) such that

\begin{enumerate}
  \item \(f(\alpha )L=Lf(\alpha ) \qquad \forall \alpha \in k,\forall L\in \End_A(V)\)
  \item \(f(\alpha )f(\beta )=f(\alpha \beta ) \qquad \forall \alpha ,\beta \in k\)
  \item \(f(\alpha )+f(\beta )=f(\alpha +\beta ) \qquad \forall \alpha ,\beta \in k\)
\end{enumerate}

Remember that the condition in 1) implies that \(f(\alpha )\) must be a scalar. Let's try the mapping \(f(\alpha )=\alpha .I\) and rewrite our conditions:

\begin{enumerate}
  \item \(\alpha .I\circ L=L\circ \alpha .I \qquad \forall \alpha \in k,\forall L\in \End_{k}(V)\)
  \item \(\alpha .I\beta .I=(\alpha \beta ).I \qquad \forall \alpha ,\beta \in k\)
  \item \((\alpha +\beta ).I=\alpha .I+\beta .I \qquad \forall \alpha ,\beta \in k\)
\end{enumerate}

All those conditions hold directly.
\end{proof}

\begin{prop}
\[
\End_A(V)\subseteq \End_{k}(V)
\]
\end{prop}

\begin{proof}
Showing that \(\End_A(V)\subseteq \End_{k}(V)\) is equivalent with showing that
\[
\phi \in \End_A(V) \Longrightarrow  \phi \in  \End_{k}(V).
\]
In other words, does \(A\)-linearity imply \(k\)-linearity ? This holds as \(k1\subseteq A\).
\end{proof}

\begin{prop}
Given a representation \(V\) of \(A\). Then \(V\) is also a representation of \(\End_A(V)\)
\end{prop}

\begin{proof}
To show that \(V\) is a representation of \(\End_A(V)\) it suffices to show that there exists a homomorphism of algebras:
\[
\rho :\End_A(V)\rightarrow \End_{k}(V).
\]
Remember that we have \(\End_A(V)\subseteq \End_{k}(V).\) So we can define \(\rho \) just as the inclusion map, which is clearly a homomorphism of algebras.
\end{proof}

\begin{defn}
Given a representation \(V\) of \(A\). Then \(V\) as representation of \(A':=\End_A(V)\) is called the \emph{centralizer module}. If we need to distinguish the two module structures on \(V\), we will write \(_AV\) to denote \(V\) as a module over \(A\), and \(_{A'}V\) to denote the centralizer module.
\end{defn}

\begin{prop}
The image of an idempotent element under an algebra homomorphism is again an idempotent element.
\end{prop}

\begin{proof}
Let \(a\in A:a^2=a\). It suffices to show that
\[
f(a)^2=f(a).
\]
This holds as:
\[
f(a)^2=f(a)f(a)=f(a^2)=f(a).
\]
\end{proof}

\begin{prop}
For any idempotent \(e\in A\) we have that

\begin{enumerate}
  \item \(\rho (e)\in \End_{k}(V)\) is idempotent
  \item if \(e\) is central, then \(\rho (e)\in \End_A(V)\)
\end{enumerate}
\end{prop}


\begin{proof}

\begin{enumerate}
  \item The statement holds as \(\rho \) is an algebra homomorphism. And remember that the image of an idempotent element under an algebra homomorphism is again an idempotent element.
  \item Showing that
\[
\rho (e)\in \End_A(V)
\]
is equivalent with showing that
\[
a.\rho (e)(v)=\rho (e)(a.v)
\]
which in turn is equivalent with
\[
\rho (a)(\rho (e)(v))=\rho (e)(\rho (a)(v)).
\]
In other notation
\[
\rho (a)\circ \rho (e)(v)=\rho (e)\circ \rho (a)(v).
\]
As \(\rho \) is an algebra homomorphism this is equivalent with
\[
\rho (ae)(v)=\rho (ea)(v).
\]
By hypothesis, we have that \(e\) is central, so \(ea=ae\) and therefore our last statement holds. 
\end{enumerate}
\end{proof}


\begin{prop}
For any \(x\in \End_A(V)\) the subspace \(\Ker(x)\subseteq V\) is a \(A\)-submodule.
\end{prop}

\begin{proof}
Showing that \(\Ker(x)\) is a subrepresentation of \(V\) is equivalent with showing that
\[
a.\Ker(x)\subseteq \Ker(x) \qquad \forall a\in A.
\]
which is equivalent with showing that
\[
x(a.v)=0 \qquad \forall a\in A,\forall v\in \Ker(x)
\]
As \(x\in \End_A(V)\), this is equivalent with showing that
\[
a.x(v)=0 \qquad \forall a\in A,\forall v\in \Ker(x) \qquad \checkmark.
\]
\end{proof}

\begin{prop}
For any \(x\in \End_A(V)\) the subspace \(\I(x)\subseteq V\) is a \(A\)-submodule.
\end{prop}

\begin{proof}
Showing that \(\I(x)\) is a subrepresentation is equivalent with showing that
\[
a.\I(x)\subseteq \I(x)
\]
which is equivalent with showing that
\[
\forall a\in A\  \forall v\in V\  \exists w\in W \qquad a.x(v)=x(w) 
\]
As \(x\in \End_A(V)\), this is equivalent with showing that
\[
\forall a\in A\forall v\in V\  \exists w\in W \qquad x(a.v)=x(w) \qquad \checkmark. 
\]
\end{proof}

\begin{prop}
If \(V=U\oplus W\) is a decomposition of \(_AV\) as direct sum of \(A\)-submodules. Then the projection \(e_U\) of \(V\) onto \(U\) along \(W\) and the projection \(e_W\) of \(V\) onto \(W\) along \(U\) satisfy:

\begin{enumerate}
  \item \(e_U,e_W\in \End_A(V)\)
  \item \(e_U\) and \(e_W\) are orthogonal idempotent elements
  \item \(1=e_U+e_W\)
\end{enumerate}

\end{prop}

Remember that if \(V=U\oplus W\), then we can regard $U,W$ as subspaces of \(V\) such that \(U\cap W=0\) and for any \(v\in V\) we have that \(v=u+w\) where \(u\in U, w\in W\). And \(a.v=a.u+a.w\) where \(a.u\in U\)

\begin{proof}[Proof of 1]
To show that \(e_U\in \End_A(V)\) it suffices to show that
\[
a.e_U(v)= e_U(a.v) \qquad \forall a\in A,v\in V
\]
which is equivalent with showing that
\[
a.e_U(u+w)= e_U(a.(u+w)) \qquad \forall a\in A,u\in U,w\in W
\]
which is equivalent with showing that
\[
a.u= e_U(a.u+a.w) \qquad \forall a\in A,u\in U,w\in W.
\]
With holds as \(U,W\) are subrepresentations.
\end{proof}

\begin{proof}[Proof of 2]
Showing that \(e_U\) is idempotent, is equivalent with showing that
\[
e_U\circ e_U=e_U
\]
which is equivalent with showing that
\[
e_U\circ e_U(v)=e_U(v) \qquad \forall v\in V
\]
which is equivalent with showing that
\[
e_U\circ e_U(u+w)=e_U(u+w) \qquad \forall u\in U,w\in W\qquad \checkmark.
\]

Showing that \(e_U\) and \(e_W\) are orthogonal is equivalent with showing that
\[
e_U\circ e_W=0 \qquad e_W\circ e_U=0
\]
which is equiavlent with showing that
\[
e_U\circ e_W(u+w)=0(u+w) \qquad e_W\circ e_U(u+w)=0(u+w) \qquad \forall u\in U,w\in W\qquad \checkmark.
\]
\end{proof}

\begin{proof}[Proof of 3]
Showing that
\[
e_U+e_W=1
\]
is equivalent with showing that
\[
e_U(u+w)+e_W(u+w)=1(u+w) \qquad \forall u\in U,w\in W \qquad \checkmark
\]
\end{proof}

\begin{prop}
If $e\in A'$ is an idempotent element then

\begin{enumerate}
  \item \(1-e\in A'\) idempotent
  \item \(1=e+(1-e)\) is a decomposition as sum of orthogonal elements
  \item \(V=eV\oplus (1-e)V\) is a decomposition of \(_AV\) as a direct sum of \(A\)-submodules
\end{enumerate}
\end{prop}


\begin{proof}[Proof of 1]
Showing that \(1-e\) is idempotent is equivalent with showing that
\[
(1-e)^2=1-e
\]
which is equivalent with showing that
\[
1-2e+e^2=1-e \qquad \checkmark.
\]
\end{proof}

\begin{proof}[Proof of 2]
To show that \(1=e+(1-e)\) is a decomposition of sum of orthogonal elements, it suffices to show that
\[
e(1-e)=0 \qquad (1-e)e=0.
\]
which is equivalent with showing that
\[
e-e^2=0. \qquad \checkmark
\]
\end{proof}

\begin{proof}[Proof of 3]
Since \(1=e+(1-e)\) is a decomposition as sum of orthogonal elements, we have that
\[
v=e(v)+(1-e)(v).
\]
Hence if \(v\in \I(e) \cap  \I(1-e)\) then
\[
v=v+v=2v.
\]
And then \(v=0\).
\end{proof}

\begin{thm}
$_AV$ is an indecomposable module iff \(1\in A'\) is a minimal idempotent
\end{thm}
