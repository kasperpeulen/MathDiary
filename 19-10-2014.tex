\section{19-10-2014}
\subsection{Representation Theory Reader Week 40}


\begin{prop}
\[
\dim_D(V)=\dim_{k}(V)/\dim_{k}(D)
\]
\end{prop}


\begin{prop}
Let \(k\) be a field, and let \(D\) bea finite dimensional \(k\)-division algebra. Then

\begin{enumerate}
  \item \(A:=\Mat_{n}(D)\) is a semisimple \(k\)-algebra
  \item with a single isomorphism class of simple \(A\)-modules, namely \(V=D^n\)
  \item \(A'=D^{\op}\)
\end{enumerate}
\end{prop}


\begin{proof}[Proof of 1]
To show that \(\Mat_{n}(D)\) is a semisimple \(k\)-algebra, it suffices to show that the regular representation is semisimple.

Showing that the regular representation is semisimple, is equivalent with showing that the regular representation is a direct sum of irreducible submodules.
It suffices to show that

\begin{enumerate}
  \item \(\Mat_{n}(D)\cong V^n \qquad V:=D^n.\)
  \item \(D^n\) is an irreducible submodule of the regular representation
\end{enumerate}

To show that \(D^n\) is irreducible submodule, it suffices to show that every nonzero vector \(v\in D^n\) is cyclic. To show that is suffices to show that
\[
\Mat_{n}(D).(d_{1},\ldots ,d_{n})=D^n \qquad \forall d_{1},\ldots ,d_{n}\in D.
\]
To show this it suffices to show that
\[
D^n\subseteq \Mat_{n}(D).(d_{1},\ldots ,d_{n}) \qquad \forall d_{1},\ldots ,d_{n}\in D.
\]
In other words, for any \(v,w\in D^n\) there exists a matrix such that
\[
v=Mw \qquad \checkmark.
\]

\end{proof}

\begin{proof}[Proof of 2]
Remember that
\[
\dim_{k}(A)\geq \sum _{i=1}^n\dim_{D_{i}}(V_{i})^2\dim_{k}(D_{i}).
\]
To show that the only irreducible representation of \(\Mat_{n}(D)\) is \(V=D^n\) it suffices to show that
\[
\dim_{D}(V)^2\dim_{k}(D)=\dim_{k}(\Mat_{n}(D))
\]
which is equivalent with showing that
\[
\dim_{D}(D^n)^2=\dim_D(\Mat_{n}(D)) \quad \checkmark.
\]
\end{proof}

\begin{proof}[Proof of 3]
\(A'=D^{op}\)

\(\Uparrow \) [by the DCP we have \(\End_{A'}(V)=\rho (A)\overset{?}{\cong }A\)]

\(\End_{D^{op}}(V)=A\)

\(\Uparrow \) [given]

\(\End_{D^{op}}(V)=\Mat_{n}(D)\)

\(\Uparrow \) [\(\End_D(D)=D^{op}\) and \(\End_{D^{op}}(D)=D\)]

\(\End_{D^{op}}(V)=\Mat_{n}(\End_{D^{op}}(D))\)

\(\Uparrow \) [by prop 4.9]

\(V\) is a finite dimensional semisimple \(D^{op}\) module and any irreducible  module is equivalent to \(D\)

\(\Uparrow \)

\(V=D^ne_{1}\oplus \cdots \oplus D^ne_{n}\) is a decomposition of \(V\) as direct sum of irreducible \(D^{op}\) modules where \({e_{1},\ldots ,e_{n}}\) is the standard \(D\)-basis of of \(V=D^n\)
\end{proof}

\begin{prop}
A finite dimensional \(k\)-algebra of the form
\[
A:=\Mat_{d_{1}}(D_{1})\oplus \cdots \oplus \Mat_{d_{n}}(D_{n})
\]
with \(D_{i}\) a finite dimensional algebra over \(k\), is a semisimple \(k\)-algebra
with \(\Irr(A)=\{V_{1},\ldots ,V_{n}\}\), where \(V_{i}=D_{i}^{d_{i}}\) is equipped with the matrix multiplication action of the summand \(\Mat_{d_{i}}(D_{i}).\)
\end{prop}

\begin{proof}
\textbf{Showing that \(A\) is a semisimple \(k\)-algebra}

\(\Uparrow \) by P4.16

The left regular representation of \(A\) is semisimple

\(\Uparrow \) direct sum of semisimple modules are semisimple

Every \(\Mat_{d_{i}}(D_{i})\) is semisimple. (by Lemma 5.1)

\textbf{Showing that $V_{i},V_{j}$ are inequivalent representations of \(A\)}

\(\Uparrow \) [by definition]

\(V_{i}=D_{i}^{d_{i}}\) and \(V_{j}=D_{j}^{d_{j}}\) are inequivalent representations of
\[
\Mat_{d_{1}}(D_{1})\oplus \cdots \oplus \Mat_{d_{n}}(D_{n})
\]

\(\Uparrow \) [\(\Mat_{d_{1}}(D_{1})=V_{1}^{d_{1}}=D_{1}^{d_{1}^2}\)]

\(V_{i}\) and \(V_{j}\) are inequivalent representations of
\[
V_{1}^{d_{1}}\oplus \cdots \oplus V_{n}^{d_{n}}
\]

\(\Uparrow \)

Let \(v_{i}\in V_{i}-0\) and \(v_{j}\in V_{j}-0\). There exists an \(M\in \Mat_{d_{i}}(D_{i})\) such that \(M(v_{i})\neq 0\), but \(M(v_{j})=0\) for any \(M\in \Mat_{d_{i}}(D_{i})\).
\end{proof}
\newpage
\begin{prop}
If \(V\) is a finite dimensional, then there exists a finite descending filtration with irreducible consecutive subquotients.
\end{prop}

\begin{proof}
There exists a finite descending filtration with irreducible consecutive subquotients.

\(\Uparrow \) by defintion

There exists submodules \(V_{0},.\ldots ,V_{n}\) such that

\(V=V_{0}\supseteq V_{1}\supseteq \ldots \supseteq V_{n}:=0 \qquad V_{i-1}/V_{i} \text{ irreducible }\)

\begin{itemize}
  \item Base case: If \(\dim_{k}(V)=0\) then \(V\) itself is already irreducible.
  \item Induction hypothesis: Suppose that the statement is true for modules of dimension smaller than $n$.
  \item Induction step: Let $V$ be a submodule of dimension $<n+1$. It suffices to show the case that $V$ is of dimension $n$. Let \(V_{1}\) be a proper submodule of $V$ of maximal dimension. Then \(V_{0}/V_{1}=V/V_{1}\) is irreducible. And by induction hypothesis we can construct a finite descending filtration now.
\end{itemize}
\end{proof}
\newpage
\begin{prop}
\(\Rad(A)\) is a two sided ideal of \(A\)
\end{prop}

\begin{proof}
\(\Uparrow \) [by definition]

The set of all elements of \(A\) which act by \(0\) in all irreducible representations of \(A\) is a two sided ideal of \(A\)

\(\Uparrow \) [by definition]

\(\Rad(A)a\subseteq \Rad(A)\) en \(a\Rad(A)\subseteq \Rad(A)\) for all \(a\in A\)

\begin{itemize}
  \item \textbf{\(\Rad(A)a\subseteq \Rad(A) \qquad \forall a\in A\)}

\(\Uparrow \)

\(ra(v)=0 \qquad \forall a\in A,\forall v\in V\subseteq \Irr(A), \forall r\in \Rad(A)\)

\(\Uparrow \) [\(a(v)\in V\) for any \(v\in V\subseteq \Irr(A)\)]

\(r(w)=0  \qquad \forall w\in V\subseteq \Irr(A), \forall r\in \Rad(A)\)

\(\Uparrow \) by definition of \(\Rad(A)\)
  \item \textbf{\(a\Rad(A)\subseteq \Rad(A)\) for all \(a\in A\)}

\(\Uparrow \)

\(ar(v)=0 \qquad \forall a\in A,\forall v\in V\subseteq \Irr(A), \forall r\in \Rad(A)\)

\(\Uparrow \)

\(a(r(v))=0 \qquad \forall a\in A,\forall v\in V\subseteq \Irr(A), \forall r\in \Rad(A)\)

\(\Uparrow \)

\(a(0)=0 \qquad \forall a\in A\)
\end{itemize}

\end{proof}

