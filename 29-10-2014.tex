\section{29-10-2014}
\subsection{Measure Theory Chapter 11}

\begin{thm}
Let \((X,\mathcal{A},\mu )\) be a measure space.

\begin{enumerate}
  \item Let \((u_{j})_{j\in \Bbb{N}}\subseteq \mathcal{L}^1(\mu )\) be an increasing sequence of integrable functions \(u_{1}\leq u_{2}\leq \ldots \) with limit \(u:=\sup_{j\in \Bbb{N}}u_{j}.\) Then \(u\in \mathcal{L}^1(\mu )\) if, and only if, \(\sup_{j\in \Bbb{N}} \int  u_{j} d\mu <\infty ,\) in which case
\[
\sup_{j\in \Bbb{N}} \int  u_{j} d\mu  = \int  \sup_{j\in \Bbb{N}} u_{j} d\mu 
\]
  \item Let \((v_{k})_{k\in \Bbb{N}}\subseteq \mathcal{L}^1(\mu )\) be a decreasing sequence of integrable functions \(v_{1}\geq v_{2}\geq  \ldots  \) with limit \(v:=\inf_{k\in \Bbb{N}}d\mu .\) Then \(v\in \mathcal{L}^1(\mu )\) if, and only if, \(\inf_{k\in \Bbb{N}}\int v_{k}\  d\mu >-\infty ,\) in which case
\[
\inf_{k\in \Bbb{N}}\int v_{k} d\mu  = \int  \inf_{k\in \Bbb{N}}v_{k}\  d\mu 
\]
\end{enumerate}
\end{thm}

\begin{thm}
Let \((X,\mathcal{A},\mu )\) be measure space and \((u_{j})_{j\in \Bbb{N}}\subseteq \mathcal{L}^1(\mu )\) be a sequence of functions such that \(|u_{j}|\leq w\) for all \(j\in \Bbb{N}\) and some \(w\in \mathcal{L}_{+}^1(\mu ).\) If \(u(x)=\lim_{j\rightarrow \infty }u_{j}(x)\) exists for almost every \(x\in X\) then \(u\in \mathcal{L}^1(\mu )\) and we have

\begin{enumerate}
  \item \(\lim_{j\rightarrow \infty } \int |u_{j}-u|\  d\mu =0\)
  \item \(\lim_{j\rightarrow \infty } \int  u_{j}\  d\mu  = \int  \lim_{j\rightarrow \infty } u_{j} d\mu  =\int  u d\mu \)
\end{enumerate}

\end{thm}

\begin{thm}
Let \(\varnothing \neq (a,b)\subseteq \Bbb{R}\) be a non-degenerate open interval and \(u:(a,b)\times X\rightarrow  \Bbb{R}\) be a function satisfying

\begin{enumerate}
  \item \(x\mapsto  u (t,x)\) is in \(\mathcal{L}^1(\mu )\) for every fixed \(t\in (a,b)\).
  \item \(t\mapsto u(t,x)\) is continuous for every fixed \(x\in X\)
  \item \(|u(t,x)|\leq w(x)\) for all \((t,x)\in (a,b)\times X\) and some \(w\in \mathcal{L}_{+}^1(\mu ).\)
\end{enumerate}

Then the function \(v:(a,b)\rightarrow \Bbb{R}\) given by
\[
t\mapsto v(t):= \int  u(t,x)\  \mu (dx)
\]
is continuous.
\end{thm}

\begin{thm}
Let \(\varnothing  \neq  (a,b) \subseteq \Bbb{R}\) be a non-degenerate open interval and \(u : (a,b) \times X \rightarrow \Bbb{R}\) be a function satsifying:

\begin{enumerate}
  \item \(x\mapsto  u(t,x)\) is in \(\mathcal{L}^1(\mu )\) for every fixed \(t\in (a,b)\).
  \item \(t\mapsto u(t,x)\) is differentiable for every fixed \(x\in X\)
  \item \(|\partial _{t}u(t,x)|\leq w(x)\) for all \((t,x)\in (a,b) \times X\) and some \(w\in \mathcal{L}_{+}^1(\mu )\)
\end{enumerate}

Then the function \(v:(a,b)\rightarrow \Bbb{R}\) given by
\[
t\mapsto  v(t):= \int  u(t,x)\  \mu (dx)
\]
is differentiable and its derivative is
\[
\partial _{t} v(t) = \int  \partial _{t} u(t,x) \mu (dx).
\]
\end{thm}

\begin{defn}
Consider on the finite interval \([a,b]\subseteq \Bbb{R}\) the partitions
\[
\pi :=\{a=t_{0}<t_{1}<\ldots <t_{k(\pi )}=b,
\]
define for a given function \(u:[a,b]\rightarrow \Bbb{R}\)
\[
m_{j}:=\inf_{x\in [t_{j-1},t_{j}]} u(x) \qquad M_{j}:=\sup_{x\in [t_{j-1},t_{j}]} u(x) \qquad j=1,2,\ldots ,k(\pi )
\]
and introduce the lower resp. upper Darboux sums
\[
S_\pi [u] := \sum _{j=1}^{k(\pi )} m_{j} (t_{j}-t_{j-1}) \qquad S^\pi [u]:= \sum _{j=1}^{k(\pi )}M_{j}(t_{j}-t_{j-1})
\]
\end{defn}
\begin{defn}
A bounded function \(u:[a,b]\rightarrow \Bbb{R}\) is said to be Riemann integrable, if the values
\[
\int _*u:=\sup_{\pi }S_\pi [u] = \inf_{\pi }S^\pi [u]:=\int ^* u
\]
(sup,inf range over all partitions \(\pi \) of \([a,b]\)) conincide and are finite. Their common value is called the Riemann integral of \(u\) and denoted by \((R)\int _{a}^b u(x)\  dx\) or \(\int _{a}^b u(x)\  dx.\)
\end{defn}

\begin{prop}
\(S_{\pi }[u]\) and \(S^\pi [u]\) correspond to simple functions \(\sigma _\pi [u]\) and \(\Sigma ^\pi [u]\) given by
\[
\sigma _\pi [u](x) = \sum _{j=1}^{k(\pi )} m_{j} 1_{[t_{j-1},t_{j})}(x) \qquad \Sigma ^\pi [u](x)= \sum _{j=1}^{k(\pi )}M_{j} 1_{[t_{j-1},t_{j})}(x)
\]
which satisfy \(\sigma _\pi [u](x)\leq u(x)\leq \Sigma ^\pi [u](x)\) and which increase resp. decrease as \(\pi \) refines.
\end{prop}

\begin{thm}
Let \(u:[a,b]\rightarrow \Bbb{R}\) be a measurable function.

\begin{enumerate}
  \item If \(u\) is Riemann integrable, then \(u\) is in \(\mathcal{L}^1(\lambda )\) and the Lebesgue and Riemann integrals coincide:
\[
\int _{[a,b]}u\  d\lambda  = (R) \int _{a}^b u(x)\  dx.
\]
  \item A bounded function $f:[a,b]\rightarrow \Bbb{R}$ is Riemann integrable if, and only if, the points \((a,b)\) where \(f\) is discontinuous are a Lebesgue null set.
\end{enumerate}
\end{thm}


\begin{defn}
An improper Riemann integral is defined as
\[
(R)\int _{0}^\infty  u(x)\  dx := \lim_{a\rightarrow \infty }(R) \int _{0}^a u(x)\  dx
\]
provided that the limit exists.
\end{defn}

\begin{thm}
Let \(u:[0,\infty ) \rightarrow \Bbb{R}\) be a measurable function which is Riemann integrable for every interval \([0,N], n\in \Bbb{N}.\) Then \(u\in \mathcal{L}^1[0,\infty )\) if, and only if,
\[
\lim_{N\rightarrow \infty }(R)\int _{0}^N |u(x)| dx <\infty .
\]
In this case, \((R)\int _{0}^\infty  u(x) dx = \int _{[0,\infty )}u\  d\lambda .\)
\end{thm}

\begin{prop}
The function \(s : (0,\infty ) \rightarrow  \Bbb{R} : x\mapsto  \frac{\sin x}{x}\) is improperly Riemann integrable but not Lebesgue integrable.
\end{prop}

\begin{prop}
Let \(f_\alpha (x):= x^\alpha , x>0\) and \(\alpha \in \Bbb{R}.\) Then
\begin{gather*}
f_\alpha \in \mathcal{L}^1(0,1)\  \Longleftrightarrow \  \alpha >-1 \\
f_\alpha  \in \mathcal{L}^1[1,\infty ) \Longleftrightarrow \  \alpha <-1
\end{gather*}
\end{prop}

\begin{prop}
The function \(f(x):=x^\alpha e^{-\beta x},x>0\) is Lebesgue integrable over \((0,\infty )\) for all \(\alpha >-1\) and \(\beta \geq 0.\)
\end{prop}

\begin{prop}
The parameter-dependent integral
\[
\Gamma (t):=\int _{(0,\infty )} x^{t-1}e^{-x}\lambda (dx) \qquad t>0
\]
is called the Gamma function. It has the following properties:

\begin{enumerate}
  \item \(\Gamma \) is continuous
  \item \(\Gamma \) is arbitrarily often differentiable
  \item \(t\Gamma (t)=\Gamma (t+1)\) in particular \(\Gamma (n+1)=n!\)
  \item \(\ln \Gamma (t)\) is convex
\end{enumerate}

\end{prop}

