\section{11-10-2014}
\subsection{Measure Theory Chapter 8}
\begin{defn}
Note that:
\(u^{-1}[a,\infty )=\{x\in X:u(x)\in [a,\infty )\}=\{x\in X:u(x)\geq a\}.\)
We define:
\[
\{u(x)\geq a\}=u^{-1}[a,\infty ).
\]
\end{defn}

\begin{thm}
Let \((X,\mathcal{A})\) be a measurable space. The function \(u:X\rightarrow \Bbb{R}\) is \(\mathcal{A}/\mathcal{B}\)-measurable if, and only if, one, hence all, of the following conditions hold

\begin{enumerate}
  \item \(\{u\geq a\}\in \mathcal{A} \qquad \forall a\in \Bbb{R}\)
  \item \(\{u>a\}\in \mathcal{A} \qquad \forall a\in \Bbb{R}\)
  \item \(\{u\leq a\}\in \mathcal{A} \qquad \forall a\in \Bbb{R}\)
  \item \(\{u<a\}\in \mathcal{A} \qquad \forall a\in \Bbb{R}\)
\end{enumerate}
\end{thm}


\begin{defn}
We define the \emph{extended real line \(\bar{\Bbb{R}}:=[-\infty ,\infty ]\)} with the following rules for all \(x\in \Bbb{R}\):
\begin{align*}
x+\infty =\infty +x=\infty  \qquad &x+-\infty =-\infty +x=-\infty  \\
\infty +\infty =\infty  \qquad &-\infty -\infty =-\infty 
\end{align*}
And for \(x\in (0,\infty ]:\)
\begin{align*}
\pm x\cdot \infty &=\infty \cdot \pm x=\pm \infty  \\
\pm x\cdot -\infty &=-\infty \cdot \pm x=\mp \infty  \\
0\cdot \pm \infty &=\pm \infty \cdot 0=0 \\
\frac{1}{\pm \infty }&=0
\end{align*}
\end{defn}

\begin{defn}
Functions which take values in \(\bar{\Bbb{R}}\) are called \emph{numerical functions}.
\end{defn}

\begin{defn}
The Borel \(\sigma \)-algebra \(\bar{\mathcal{B}}=\mathcal{B}(\bar{\Bbb{R}})\) is defined by:
\[
\bar{\mathcal{B}}:=\bigg\{B\cup S : B\in \mathcal{B}\space  \text{ and }\space S\in \Big\{\varnothing ,\{-\infty \},\{\infty \},\{-\infty ,\infty \}\Big\}\bigg\}
\]
\end{defn}

\begin{thm}
We have \(\mathcal{B}(\Bbb{R})=\Bbb{R}\cap \mathcal{B}(\bar{\Bbb{R}})\). Moreover \(\bar{\mathcal{B}}\) is generated by all sets of the form \([a,\infty ]\) or \((a,\infty ]\) or \([-\infty ,a)\) or \([-\infty ,a]\) where \(a\in \Bbb{R}\)
\end{thm}

\begin{defn}
Let \((X,\mathcal{A})\) be a measurable space. We write \(\mathcal{M}:=\mathcal{M}(\mathcal{A})\) and \(\mathcal{M}_{\bar{\Bbb{R}}}:=\mathcal{M}_{\bar{\Bbb{R}}}(\mathcal{A})\) for the families of real valued \(\mathcal{A}/\mathcal{B}\)-measurable and numerical \(\mathcal{A}/\bar{\mathcal{B}}\)-measurable functions on \(X\).
\end{defn}

\begin{defn}
A \emph{simple function} \(g:X\rightarrow \Bbb{R}\) on a measurable space \((X,\mathcal{A})\) is a function of the form
\[
g(x):=\sum _{j=1}^M y_{j}\mathbf{1}_{A_{j}}(x)
\]
with finitely many sets \(A_{1},\ldots ,A_{m}\in \mathcal{A}\) and \(y_{1},\ldots ,y_M\in \Bbb{R}\). The set of simple functions is denoted by \(\mathcal{E}\) or \(\mathcal{E}(\mathcal{A})\).

If the sets \(A_{1},\ldots ,A_M\) are mutally disjoint we call
\[
\sum _{j=0}^M y_{j}\mathbf{1}_{A_{j}}(x)
\]
with \(y_{0}:=0\) and \(A_{0}:=(A_{1}\cup \ldots \cup A_M)^c\) a \emph{standard representation} of \(g\). Caution, this representation is not unique.
\end{defn}

\begin{thm}
Let \((X,\mathcal{A})\) be a measurable space. Every \(\mathcal{A}/\bar{\mathcal{B}}\)-measurable numerical function \(u:X\rightarrow \bar{\Bbb{R}}\) is the pointwise limit of simple functions:
\[
u(x)=\lim_{j\rightarrow \infty }f_{j}(x)
\]
where \(f_{j}\in \mathcal{E}(\mathcal{A})\) and \(|f_{j}|\leq |u|.\)

If \(u\geq 0\), all \(f_{j}\) can be chosen to be positive and increasing towards \(u\) so that \(u=\sup_{j\in \Bbb{N}}f_{j}\).
\end{thm}

\begin{thm}
Let \((X,\mathcal{A})\) be a measurable space. If \(u_{j}:X\rightarrow \bar{\Bbb{R}}, j\in \Bbb{N}\) are measurable functions, then so are
\[
\sup_{j\in \Bbb{N}}u_{j} \quad \inf_{j\in \Bbb{N}}u_{j} \quad \limsup_{j\rightarrow \Bbb{N}}u_{j}\quad \liminf_{j\rightarrow \Bbb{N}}u_{j}
\]
and whenever it exists
\[
\lim_{j\rightarrow \infty }u_{j}.
\]
\end{thm}

\begin{thm}
Let \(u,v\) be \(\mathcal{A}/\bar{\mathcal{B}}\)-measurable functions. Then the functions
\[
u\pm v \quad uv\quad u\vee v:=\max\{u,v\}\quad u\wedge v:=\min\{u,v\}
\]
are \(\mathcal{A}/\bar{\mathcal{B}}\)-measurable (whenever they are defined).
\end{thm}

\begin{thm}
A function \(u\) is \(\mathcal{A}/\mathcal{B}\) measurable if, and only if, \(u^\pm \) are \(\mathcal{A}/\bar{\mathcal{B}}\) measurable.
\end{thm}

\begin{thm}
Let \(T:(X,\mathcal{A})\rightarrow (X',\mathcal{A}')\) be an \(\mathcal{A}/\mathcal{A}'\)-measurable map and let \(\sigma (T)\subseteq \mathcal{A}\) be the \(\sigma \)-algebra generated by \(T\). Then \(u=w(T)\) for some \(\mathcal{A}'/\bar{\mathcal{B}}\) measurable function \(w:X'\rightarrow \bar{\Bbb{R}}\) if and only if
\(u:X\rightarrow \bar{\Bbb{R}}\) is \(\sigma (T)/\bar{\mathcal{B}}\)-measurable.
\end{thm}

\begin{prop}
Let \((X,\mathcal{A})\) be a measurable space. We define the indicator function:
\[
1_A : X\rightarrow  \Bbb{R} : x\in A \mapsto 1 \quad  x\in X-A\mapsto 0
\]
Show that the indicator function is measurable if, and only if, \(A\in \mathcal{A}.\)
\end{prop}

\begin{proof}
To show that \(1_A\) is measurable, it suffices to show that
\[
1_A^{-1}(a,\infty )\in \mathcal{A}.
\]

Note that
\[
1_A^{-1}(a,\infty )=\{x\in X:1_A(x)\in (a,\infty )\}=\{1_A >a\}
\]
If \(a\geq 1\), then \(1_A^{-1}(a,\infty )=\varnothing \).

If \(a\in [0,1)\), then  \(1_A^{-1}(a,\infty )=A.\)

If \(a<0\), then \(1_A^{-1}(a,\infty )=X\).
\end{proof}

\begin{prop}
Let \(A_{1},\ldots ,A_M\in \mathcal{A}\) be mutally disjoint sets and \(y_{1},\ldots ,y_M\in \Bbb{R}\). Then the function
\[
g: X\rightarrow \Bbb{R} : x\mapsto \sum _{j=1}^M y_{j}1_{A_{j}}(x)
\]
is measurable.
\end{prop}

\begin{proof}
To show that \(g\) is measurable it suffices to show that
\[
\{g>a\}\in \mathcal{A}
\]
i.e.
\[
\Big\{x\in X:\sum _{j=1}^M y_{j}1_{A_{j}}(x)>a\Big\}=\bigcup _{j:y_{j}>a}A_{j}\in \mathcal{A}.
\]
\end{proof}

\begin{thm}[Problem 8.3i]
Let \((X,\mathcal{A})\) be a measurable space. Let \(f,g : X\rightarrow \Bbb{R}\) be measurable functions. Show that for every \(A\in \mathcal{A}\) the functions \(h(x):=f(x)\) if \(x\in A\) and $h(x):=g(x)$, if \(x\not\in A\), is measurable.
\end{thm}

\begin{proof}
Note that
\[
h(x):=1_A(x)f(x)+1_{A^c}(x)g(x).
\]

And remember that sums and products of measurable functions are again measurable.
\end{proof}

\begin{thm}[Problem 8.3ii]
Let \((f_{j})_{j\in \Bbb{N}}\) be a sequence of measurable functions and let \((A_{j})_{j\in \Bbb{N}}\subseteq \mathcal{A}\) such that \(\bigcup _{j\in \Bbb{N}}A_{j}=X.\) Suppose that \(f_{j}|_{A_{j}\cap A_{k}}=f_{k}|_{A_{j}\cap A_{k}}\) for all \(j,k\in \Bbb{N}\) and set \(f(x):=f_{j}(x)\) if \(x\in A_{j}\). Show that \(f:X\rightarrow \Bbb{R}\) is measurable.<br />
\end{thm}

\begin{proof}
We have that:
\[
f^{-1}(B)=\bigcup _{j\in \Bbb{N}}A_{j}\cap f^{-1}(B)=\bigcup _{j\in \Bbb{N}}A_{j}\cap f_{j}^{-1}(B)\in \mathcal{A}
\]
\end{proof}

\begin{thm}[Problem 8.4]
Let \((X,\mathcal{A})\) be a measurable space and let \(\mathcal{B}\subset \mathcal{A}\) be a sub-\(\sigma \)-algebra. Show that \(\mathcal{M}(\mathcal{B})\subset \mathcal{M}(\mathcal{A}).\)
\end{thm}

\begin{proof}
To show that
\[
\mathcal{M}(\mathcal{B})\subset \mathcal{M}(\mathcal{A})
\]
it suffices to show there exists a \(\mathcal{A}\)-measurable function that is not \(\mathcal{B}\)-measurable. By hypothesis, we have an element \(A\in \mathcal{A}\), that is not in \(\mathcal{B}\), i.e. \(A\not\in \mathcal{B}\). Since \(1_A\) is \(\mathcal{B}\)-measurable if, and only if, \(B\in \mathcal{B}\), we have find the \(\mathcal{A}\)-measurable function where we where looking for.
\end{proof}

\begin{thm}
Let \(u:\Bbb{R}\rightarrow \Bbb{R}\) be differentiable. Explain why \(u\) and \(u'=du/dx\) are measurable.
\end{thm}

\begin{proof}
If \(u\) is differentiable, it is continuous, hence measurable. Since \(u'\) exists, we can write it in the form
\[
u'(x)=\lim_{k\rightarrow \infty } \frac{u(x+1/k)-u(x)}{1/k}
\]
i.e. as limit of measurable functions. Thus, \(u'\) is also measurable.
\end{proof}

\begin{thm}[Problem 8.17]
Show that the measurability of \(|u|\) does not, in general, imply the measurability of \(u\).
\end{thm}

\begin{proof}
Let \(A\subseteq \Bbb{R}\) be such that \(A\not\in \mathcal{B}\). Then it is clear that
\[
u(x):=1_A(x)-1_{A^c}(x)
\]
is not measurable. Take
\[
\{u=1\}=A\not\in \mathcal{A}.
\]
But \(|u(x)|=1\), which is a continuous function and therefore measurable.

\end{proof}

\begin{thm}[Problem 8.14]
Consider \((\Bbb{R},\mathcal{B})\) and \(u:\Bbb{R}\rightarrow \Bbb{R}\). Show that \(\{x\}\in \sigma (u)\) for all \(x\in \Bbb{R}\) if, and only if, \(u\) is injective.
\end{thm}

\begin{proof}
To show that \(u\) is injective, it suffices to assume $x,y\in \Bbb{R}$ and show that
\[
u(x)=u(y)\Longrightarrow x=y.
\]
Showing that is equivalent with showing that
\[
|\{u=u(x_{0})\}|=1.
\]
We surely have that \(\{x_{0}\}\subseteq \{u=u(x_{0})\}\). And note that
\[
\{x_{0}\}\in \sigma (u)=\sigma (u^{-1}(\mathcal{B}))
\]
just means that \(\{x_{0}\}=u^{-1}(B)\) for some $B\in \mathcal{B}$.
\end{proof}

\begin{proof}
Assume that \(u\) is injective,. This means that every point in the range \(u(\Bbb{R})\) comes exactly from unique defined \(x\in \Bbb{R}\). This can be expressed by saying that \(\{x\}=u^{-1}(\{u(x)\})=\{u(x)\}.\) But then
\[
\{x\}\in \sigma (u)=\sigma (u^{-1}(\mathcal{B})).
\]
\end{proof}

