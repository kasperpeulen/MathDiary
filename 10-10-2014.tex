\section{10-10-2014}
\begin{defn}
Let \((X,\mathcal{A}),(X',\mathcal{A}')\) be two measurable spaces. A map \(T:X\rightarrow X'\) is called \emph{\(\mathcal{A}/\mathcal{A}'\)-measureable} (or \emph{measurable} unlesss this is too amiguous) if the pre-image of every measurable set is a measurable set:
\[
T^{-1}(A')\in \mathcal{A} \qquad \forall A'\in \mathcal{A}'.
\]
We often denote this by \(T^{-1}(\mathcal{A}')\subseteq \mathcal{A}'\).
\end{defn}

\begin{defn}
A \emph{random variable} is a measurable map from a probability space (i.e. \(\mu (X)=1)\) to any measurable space.
\end{defn}

\begin{thm}[Lemma 7.2]
Let \((X,\mathcal{A}),(X',\mathcal{A}')\) be measurable spaces and let \(\mathcal{A}'= \sigma (\mathcal{G}').\) Then \(T:X\rightarrow X'\) is \(\mathcal{A}/\mathcal{A}'\)-measurable if and only if
\[
T^{-1}(G')\in \mathcal{A} \qquad \forall G'\in \mathcal{G}'.
\]
\end{thm}

\begin{thm}[Problem 7.1]
Show that
\[
\tau _{x}:\Bbb{R}^n \rightarrow \Bbb{R}^n : B\mapsto B-x 
\]
is a \(\mathcal{B}(\Bbb{R}^n)/\mathcal{B}(\Bbb{R}^n)\) measurable map.
\end{thm}

\begin{proof}
Showing that
\[
\tau _{x}:\mathcal{B}(\Bbb{R}^n) \rightarrow \mathcal{B}(\Bbb{R}^n) : B\mapsto B-x
\]
is \(\mathcal{B}(\Bbb{R}^n)/\mathcal{B}(\Bbb{R}^n)\) measurable, is equivalent with showing that
\[
\tau _{x}^{-1}(B)\in \mathcal{B}(\Bbb{R}^n) \qquad \forall B\in \mathcal{B}(\Bbb{R}^n),
\]
which in turn is equivalent with showing that
\[
x+B\in \mathcal{B}(\Bbb{R}^n) \qquad \forall B\in \mathcal{J}(\Bbb{R}^n),
\]
which in turn is equivalent with showing that
\[
x+[a,b)\in \mathcal{B}(\Bbb{R}^n) \qquad \forall a,b\in \Bbb{R}^n.
\]
This follows as \(x+[a,b)=[x+a,x+b)\in \mathcal{J}(\Bbb{R}^n)\subseteq \mathcal{B}(\Bbb{R}^n).\)
\end{proof}

\begin{thm}
Every continuous map \(T:\Bbb{R}^n\rightarrow \Bbb{R}^m\) is \(\mathcal{B}^n/\mathcal{B}^m\) measurable.
\end{thm}

\begin{proof}
Showing that
\[
T:\Bbb{R}^n\rightarrow \Bbb{R}^m
\]
is \(\mathcal{B}^n/\mathcal{B}^m\) measurable, is equivalent with showing that
\[
T^{-1}(\mathcal{O}^m)\subseteq \mathcal{B}^n.
\]
As \(\mathcal{O}^n\subseteq \sigma (\mathcal{O}^n)= \mathcal{B}^n\), it suffices to show that
\[
T^{-1}(\mathcal{O}^m)\subseteq \mathcal{O}^n,
\]
which follows from the continuity of \(T\).
\end{proof}

\begin{defn}
Let \((T_{i})_{i\in I}\) be arbitrarily many mappings \(T_{i}:X\rightarrow X_{i}\) from the same space \(X\) into measurable spaces \((X_{i},\mathcal{A}_{i})\). The smallest \(\sigma \)-algebra on \(X\) that makes all \(T_{i}\) simultaneously measurable is
\[
\sigma (T_{i}:i\in I):=\sigma \Big(\bigcup _{i\in I}T_{i}^{-1}(\mathcal{A}_{i})\Big).
\]
We say that \(\sigma (T_{i}:i\in I)\) is \emph{generated by the family \((T_{i})_{i\in I}\)}.
\end{defn}

\begin{thm}
Let \((X_{j},A_{j}),j=1,2,3\), be measurable spaces and \(T:X_{1}\rightarrow X_{2},S:X_{2}\rightarrow X_{3}\) be \(\mathcal{A}_{1}/\mathcal{A}_{2}-\) resp. \(\mathcal{A}_{2}/\mathcal{A}_{3}\)-measurable maps. Then \(S\circ T:X_{1}\rightarrow X_{3}\) is \(\mathcal{A}_{1}/\mathcal{A}_{3}\)-measurable.
\end{thm}
\newpage
\begin{thm}[Problem 7.4]
Let \(X\) be a set, \((X_{i},\mathcal{A}_{i}),\space i\in I\),be arbitrarily many measurable spaces, and \(T_{i}:X\rightarrow X_{i}\) be a family of maps. Show that a map \(f\) from a measurable space \((F,\mathcal{F})\) to \((X,\sigma (T_{i}:i\in I)\) is measurable if, and only if, all maps \(T_{i}\circ f\) are \(\mathcal{F}/\mathcal{A}_{i}\)-measurable.
\end{thm}

\begin{proof}[Proof of $\Longrightarrow$]
To show that all maps \(T_{i}\circ f\) are \(\mathcal{F}/\mathcal{A}_{i}\)-measurable, it suffices to show that \(T_{i}:X\rightarrow X_{i}\) is \(\sigma (T_{i}:i\in I)/\mathcal{A}_{i}\)-measurable and \(f:F\rightarrow X\) is \(\mathcal{F}/\sigma (T_{i}:i\in I)\)-measurable.

By hypothesis, is suffices to show that \(T_{i}:X\rightarrow X_{i}\) is \(\sigma (T_{i}:i\in I)/\mathcal{A}_{i}\)-measurable, which is equivalent with showing that
\[
T_{i}^{-1}(A_{i})\in \sigma (T_{i}:i\in I)\qquad \forall A_{i}\in \mathcal{A}_{i}.
\]
It suffices to assume \(A_{i}\in \mathcal{A}_{i}\) and show that
\[
T_{i}^{-1}(A_{i})\in \bigcup _{i\in I}T_{i}^{-1}(\mathcal{A}_{i}) \qquad \checkmark.
\]
\end{proof}

\begin{proof}[Proof of $\Longleftarrow$]
To show that a map \(f\) from a measurable space \((F,\mathcal{F})\) to \((X,\sigma (T_{i}:i\in I)\) is measurable, it suffices to show that
\[
f^{-1}(\bigcup _{i\in I}T_{i}^{-1}(\mathcal{A}_{i}))\subseteq \mathcal{F}
\]
To show this it suffices to show that
\[
\bigcup _{i\in I}f^{-1}(T_{i}^{-1}(\mathcal{A}_{i}))\subseteq \mathcal{F},
\]
to show this it suffices to show that
\[
f^{-1}(T_{i}^{-1}(\mathcal{A}_{i}))\subseteq \mathcal{F},
\]
to show this it suffices to show that
\[
(T_{i}\circ f)^{-1}(\mathcal{A}_{i})\subseteq \mathcal{F}.
\]
This follows by hypothesis.
\end{proof}

\begin{thm}[Problem 7.8]
Let \(T:X\rightarrow Y\) be any map. Show that
\[
T^{-1}(\sigma (\mathcal{G)})=\sigma (T^{-1}(\mathcal{G}))
\]
holds for arbitrary families of \(\mathcal{G}\) of subsets of \(Y\).
\end{thm}

\begin{proof}
To show that
\[
T^{-1}(\sigma (\mathcal{G}))=\sigma (T^{-1}(\mathcal{G}))
\]
it suffices to show:

\begin{enumerate}
  \item \(T^{-1}(\sigma (\mathcal{G)})\subseteq \sigma (T^{-1}(\mathcal{G}))\)
  \item \(\sigma (T^{-1}(\mathcal{G}))\subseteq T^{-1}(\sigma (\mathcal{G}))\)
\end{enumerate}

To show
\[
T^{-1}(\sigma (\mathcal{G}))\subseteq \sigma (T^{-1}(\mathcal{G})),
\]
it suffices to show that
\(T\) is \(\sigma (T^{-1}(\mathcal{G}))/\sigma (\mathcal{G})\) measurable.

To show that it suffices to show that
\[
T^{-1}(\mathcal{G})\subseteq \sigma (T^{-1}(\mathcal{G})) \qquad \checkmark.
\]

To show
\[
\sigma (T^{-1}(\mathcal{G}))\subseteq T^{-1}(\sigma (\mathcal{G})),
\]
it suffices to show that
\[
T^{-1}(\mathcal{G})\subseteq T^{-1}(\sigma (\mathcal{G})) \qquad \checkmark.
\]
\end{proof}

\begin{defn}
A family \(\mathcal{D}\subseteq \mathcal{P}(X)\) is a \emph{Dynkin system} if
\begin{gather*}
X\in \mathcal{D} \\
D\in \mathcal{D}\space \Longrightarrow \space D^c\in \mathcal{D} \\
(D_{j})_{j\in \Bbb{N}}\subseteq \mathcal{D} \text{ pairwise disjoint }\space \Longrightarrow \space \bigcup _{j\in \Bbb{N}}D_{j}\in \mathcal{D}
\end{gather*}
\end{defn}

\begin{defn}
Let \(\mathcal{G}\subseteq \mathcal{P}(X).\) Then there is a smallest Dynkin system \(\delta (\mathcal{G})\) containing \(\mathcal{G}\). \(\delta (\mathcal{G})\) is called the \emph{Dynkin system generated by \(\mathcal{G}\)}.
\end{defn}

\begin{prop}
Show that
\[
\mathcal{G}\subseteq \delta (\mathcal{G})\subseteq \sigma (\mathcal{G}).
\]
\end{prop}

\begin{proof}
We have that \(\mathcal{G}\subseteq \sigma (\mathcal{G})\). And therefore \(\delta (\mathcal{G})\subseteq \delta (\sigma (\mathcal{G}))=\sigma (\mathcal{G}).\)
\end{proof}

\begin{thm}
A Dynkin system \(\mathcal{D}\) is a \(\sigma \)-algebra if, and only if, it is stable under finite intersections: \(D,E\in \mathcal{D}\space \Longrightarrow \space D\cap E\in \mathcal{D}\)
\end{thm}

\begin{proof}
It suffices to show that a \(\cap \)-stable Dynkin system is stable under countable unions. To show this, it suffices to show that given $(D_{j})_{j\in \Bbb{N}}\in \mathcal{D}$, we have
\[
D:=\bigcup _{j\in \Bbb{N}}D_{j}\in \mathcal{D}.
\]

Set \(E_{1}=D_{1}\in \mathcal{D}\). And \(E_{2}:=D_{2}\cap D_{1}^c.\) And \(E_{3}=D_{3}\cap D_{2}^c\cap D_{1}^c\). And so on. Then
\[
D=\bigcup _{j\in \Bbb{N}}E_{j}\in \mathcal{D}.
\]
\end{proof}

\begin{thm}
If \(\mathcal{G}\subseteq \mathcal{P}(X)\) is stable under finite intersections, then \(\delta (\mathcal{G})=\sigma (\mathcal{G}).\)
\end{thm}

\begin{proof}
It suffices to show that \(\sigma (\mathcal{G})\subseteq \delta (\mathcal{G})\). As \(\mathcal{G}\subseteq \delta (\mathcal{G})\) it suffices to show that \(\delta (\mathcal{G})\) is a \(\sigma \)-algebra. To show that \(\delta (\mathcal{G})\) is a \(\sigma \)-algebra, it suffices to show that \(\delta (\mathcal{G})\) is stable under finite intersections.

Fix \(D\in \delta (G)\). Consider \(\mathcal{D}_D:=\{Q\subseteq X : Q\cap D \in  \delta (\mathcal{G}) \}\). It suffices to show that \(\delta (\mathcal{G})\subseteq \mathcal{D}_D\). To show that it suffices to show that \(\mathcal{D}_D\) is a Dynkin system and that \(\mathcal{G}\subseteq \mathcal{D}_D\).

To show that \(\mathcal{G}\subseteq \mathcal{D}_D\), it suffices to show that
\[
G\cap D\in \delta (\mathcal{G}) \qquad \forall G\in \mathcal{G},
\]
to show that it suffices to show that
\[
\delta (\mathcal{G})\subseteq \mathcal{D}_G \qquad \forall G\in \mathcal{G},
\]
to show that it suffices to show that (as \(\mathcal{D}_G\) is a dynkin system)
\[
\mathcal{G}\subseteq \mathcal{D}_G \qquad \forall G\in \mathcal{G}.
\]
This follows from \(\mathcal{G}\subseteq \delta (\mathcal{G})\) and \(\mathcal{G}\) is $\cap $-stable.
\end{proof}

\begin{prop}
\[
A_{j}\uparrow A\space \Longrightarrow \space A_{j}\cap B\uparrow A\cap B
\]
\end{prop}

\begin{proof}
To show that
\[
A_{j}\uparrow A\space \Longrightarrow \space A_{j}\cap B\uparrow A\cap B,
\]
it suffices to show that
\[
A=\bigcup _{j}A_{j}\space \Longrightarrow \space A\cap B=\bigcup _{j}A_{j}\cap B,
\]
which is equivalent with showing that
\[
\bigg(\bigcup _{j}A_{j}\bigg)\cap B=\bigcup _{j}A_{j}\cap B \qquad \checkmark.
\]

\end{proof}

\begin{defn}
An \emph{exhausting sequence} \((A_{j})_{j\in \Bbb{N}}\subseteq \mathcal{A}\) is an increasing sequence of sets \(A_{1}\subseteq A_{2}\subseteq A_{3}\subseteq ...\) such that \(\bigcup _{j\in \Bbb{N}}A_{j}=X\).
\end{defn}
\newpage
\begin{thm}
Assume that \((X,\mathcal{A})\) is a measurable space and that \(\mathcal{A}=\sigma (\mathcal{G})\) is generated by a family \(\mathcal{G}\) such that

\begin{itemize}
  \item \(\mathcal{G}\) is stable under finite intersections \(G,H\in \mathcal{G}\space \Longrightarrow \space G\cap H\in \mathcal{G}\)
  \item there exists an exhausting sequence \((G_{j})_{j\in \Bbb{N}}\subseteq \mathcal{G}\) with \(G_{j}\uparrow X\)
\end{itemize}

Any two measure \(\mu ,\nu \) that coincide on \(\mathcal{G}\) and are finite for all members of the exhausting sequence \(\mu (G_{j})=\nu (G_{j})<\infty \), are equal on \(\mathcal{A}\), i.e.
\[
\mu (A)=\nu (A) \qquad \forall A\in \mathcal{A}.
\]
\end{thm}

\begin{proof}
Remember that for any increasing sequence \((A_{j})_{j\in \Bbb{N}}\subseteq \mathcal{A}\) with \(A_{j}\uparrow A\in \mathcal{A}\) we have
\[
\mu (A)=\lim_{j\in \infty }\mu (A_{j}).
\]
To show that
\[
\mu (A)=\nu (A) \qquad \forall A\in \mathcal{A}
\]
it suffices to show that (as \(G_{j}\cap A\uparrow X\cap A)\)
\[
\lim_{j\in \infty }\mu (G_{j}\cap A)=\lim_{j\in \infty }\mu (G_{j}\cap A) \qquad \forall A\in \mathcal{A}
\]
To show that it suffices to show that
\[
\mu (G_{j}\cap A)=\nu (G_{j}\cap A) \qquad \forall j\in \Bbb{N}, \quad \forall A\in \mathcal{A}.
\]

Consider \(\mathcal{D}_{j}:=\{A\in \mathcal{A} : \mu (G_{j}\cap A)=\nu (G_{j}\cap A)\}\). It suffices to show that \(\mathcal{A}\subseteq \mathcal{D}_{j}\), which is equivalent with showing \(\sigma (\mathcal{G})\subseteq \mathcal{D}_{j}\).

As \(\mathcal{G}\) is stable under finite intersections, it suffices to show that \(\delta (\mathcal{G})\subseteq \mathcal{D}_{j}.\)

As \(\mathcal{G}\) is stable under finite intersections and \(\mu (\mathcal{G})=\nu (\mathcal{G})\), we have that \(\mathcal{G}\subseteq D_{j}\) and therefore it suffices to show that \(\mathcal{D}_{j}\) is a Dynkin system.

Which you can check.
\end{proof}

\begin{thm}
The \(n\)-dimensional Lebesgue measure \(\lambda ^n\) is invariant under translations, i.e.
\[
\lambda ^n(x+B)=\lambda ^n(B) \qquad\forall x\in \Bbb{R}^n,\forall B\in \mathcal{B}(\Bbb{R}^n).
\]
\end{thm}

\begin{proof}
Set \(\nu (B):=\lambda ^n(x+B)\) for some fixed \(x\in \Bbb{R}^n\). It suffices to show that
\[
\lambda ^n(B)=\nu (B)\qquad B\in \mathcal{B}.
\]
To show that, it suffices to show that

\begin{enumerate}
  \item \(\mathcal{J}\) is \(\cap \)-stable $\quad \checkmark$
  \item \(\mathcal{J}\) admits an exhausting sequence
\begin{itemize}
  \item \([-j,j)\uparrow \Bbb{R}^n\) \(\quad \checkmark\)
\end{itemize}
  \item \(\lambda ^n|_\mathcal{J}=\nu |_\mathcal{J}\)
\begin{align*}
v([a,b))&=\lambda ^n[x+a,x+b) \\
&=\lambda ^n[a,b)
\end{align*}
  \item \(\nu \) is a measure on \(\mathcal{B}^n\)
\end{enumerate}

To show that \(\nu \) is a measure on \(\mathcal{B}^n\), it suffices to show that
\[
\nu (\bigcup _{j\in \Bbb{N}}B_{j})=\sum _{j\in \Bbb{N}}\nu (B_{j}),
\]
which is equivalent with
\[
\lambda ^n(x+\bigcup _{j\in \Bbb{N}}B_{j})=\sum _{j\in \Bbb{N}}\lambda ^n(x+B_{j}).
\]
It suffices to show
\[
B\in \mathcal{B}^n\space \Longrightarrow \space x+B\in \mathcal{B}^n.
\]

Which we have already proven.

\end{proof}

\begin{thm}
Let \((X,\mathcal{A}),(X,\mathcal{A}')\) be measurable spaces and \(T:X\rightarrow X'\) be an \(\mathcal{A}/\mathcal{A}'\) measurable map. For every measure \(\mu \) on \((X,\mathcal{A})\),
\[
\mu '(A'):=\mu (T^{-1}(A')), \qquad A'\in \mathcal{A}'.
\]
The measure \(\mu '\) is called the \emph{image measure} of \(\mu \) under \(T\) and is denoted by \(T\circ \mu \) or \(\mu \circ T^{-1}\).
\end{thm}

\begin{thm}[Problem 7.7]
Use image measures to give a new proof of Problem 5.8, i.e. to show that
\[
\lambda ^n(t\cdot B)=t^n\lambda ^n(B) \qquad \forall B\in \mathcal{B}(\Bbb{R}^n),\forall t>0
\]
\end{thm}

\begin{proof}
Set \(\nu (B):=t^n\lambda ^n(B)\) for some fixed \(x\in \Bbb{R}^n\). It suffices to show that
\[
\lambda ^n(tB)=\nu (B)\qquad \forall B\in \mathcal{B}.
\]
To show that, it suffices to show that

\begin{enumerate}
  \item \(\mathcal{J}\) is \(\cap \)-stable $\quad \checkmark$
  \item \(\mathcal{J}\) admits an exhausting sequence
\begin{itemize}
  \item \([-j,j)\uparrow \Bbb{R}^n\) \(\quad \checkmark\)
\end{itemize}
  \item \(\lambda ^n|_\mathcal{J}=\nu |_\mathcal{J}\)
\begin{align*}
\nu ([a,b))&=\lambda ^n[ta,tb) \\
&=t^n\lambda ^n[a,b)
\end{align*}
  \item \(\nu \) is a measure on \(\mathcal{B}^n\) as it is a composition of the inverse of a  measurable map and a measure.
\end{enumerate}
\end{proof}




