\section{12-10-2014}
\subsection{Measure Theory Chapter 9}
\begin{defn}
Let \(f=\sum _{j=0}^M y_{j}1_{A_{j}}\in \mathcal{E}^+\) be a simple function in standard representation. Then the number
\[
I_\mu (f):=\sum _{j=0}^My_{j}\mu (A_{j})\in [0,\infty ]
\]
is called the (\(\mu \)-)integral of \(f\).
\end{defn}

\begin{thm}

\begin{enumerate}
  \item \(I_\mu (1_A)=\mu (A) \qquad \forall A\in \mathcal{A}\)
  \item \(I_\mu (\lambda f)=\lambda I_\mu (f) \qquad \forall \lambda \geq 0\)
  \item \(I_\mu (f+g)=I_\mu (f)+I_\mu (g)\)
  \item \(f\leq g\space \Longrightarrow \space I_\mu (f)\leq I_\mu (g)\)

\end{enumerate}
\end{thm}

\begin{defn}
Let \((X,\mathcal{A},\mu )\) be a measure space. The (\(\mu \)-)integral of a positive numerical function \(u\in \mathcal{M}^+_{\bar{\Bbb{R}}}\) is given by
\[
\int  u d\mu :=\sup\{I_\mu (g) : g\leq u, g\in \mathcal{E}^+\}\in [0,\infty ].
\]
If we need to emphasize the integration variable, we also write
\[
\int u(x)\mu (dx)\quad \text{ or }\quad \int u(x)d\mu (x)
\]
\end{defn}

\begin{thm}
For all \(f\in \mathcal{E}^+\) we have \(\int fdu=I_\mu (f).\)
\end{thm}

\begin{thm}
Let \((X,A,\mu )\) be a measure space. For an increasing sequence of numerical functions \((u_{j})_{j\in \Bbb{N}}\subseteq \mathcal{M}^+_{\bar{\Bbb{R}}}, 0\leq u_{j}\leq u_{j+1}\leq \ldots \), we have \(u:=\sup_{j\in \Bbb{N}}u_{j}\in \mathcal{M}_{\bar{\Bbb{R}}}^+\) and
\[
\int \sup_{j\in \Bbb{N}}u_{j}d\mu  = \sup_{j\in \Bbb{N}} \int u_{j} d\mu 
\]
\end{thm}

\begin{thm}
Let \(u\in \mathcal{M}_{\bar{\Bbb{R}}}^+\). Then
\[
\int ud\mu  = \lim_{j\rightarrow \infty }\int f_{j}d\mu 
\]
holds for every increasing sequence \((f_{j})_{j\in \Bbb{N}}\subseteq \mathcal{E}^+\) with \(\lim_{j\rightarrow \infty }f_{j}=u\).
\end{thm}

\begin{thm}
Let \(u,v\in \mathcal{M}^+_{\bar{\Bbb{R}}}\). Then

\begin{enumerate}
  \item \(\int  1_A d\mu  = \mu (A) \quad \forall A\in \mathcal{A}\)
  \item \(\int  \alpha u d\mu  = \alpha  \int u d\mu \)
  \item \(\int (u+v)d\mu  = \int ud\mu  +\int vd\mu \)
  \item \(u\leq v\space \Longrightarrow \space \int ud\mu \leq \int vd\mu \)
\end{enumerate}
\end{thm}


\begin{thm}
Let \((u_{j})_{j\in \Bbb{N}}\subseteq \mathcal{M}^+_{\bar{\Bbb{R}}}\). Then \(\sum _{j=1}^\infty u_{j}\) is measurable and we have
\[
\int \sum _{j=1}^\infty u_{j}d\mu =\sum _{j=1}^\infty \int u_{j}d\mu 
\]
\end{thm}

\begin{thm}
Let \((u_{j})_{j\in \Bbb{N}}\subseteq \mathcal{M}^+_{\bar{\Bbb{R}}}\) be a sequence of positive measurable numerical functions. Then \(u:=\liminf_{j\in \infty }\int u_{j}d\mu \) is measurable and
\[
\int \liminf_{j\rightarrow \infty }u_{j}d\mu \leq \liminf_{j\rightarrow \infty }\int u_{j}d\mu 
\]
\end{thm}

\begin{thm}[Problem 9.1]
Let \(f:X\rightarrow \Bbb{R}\) be a positive simple function of the form
\[
f(x)=\sum _{j=1}^m \xi _{j}1_{A_{j}}(x)\qquad \xi _{j}\geq 0 , A_{j}\in \mathcal{A}.
\]
Show that
\[
I_\mu (f)=\sum _{j=1}^m\xi _{j}\mu (A_{j})
\]
\end{thm}

\begin{proof}
\[
I_\mu (f)=I_\mu \bigg(\sum _{j=1}^m \xi _{j}1_{A_{j}}\bigg)=\sum _{j=1}^m\xi _{j}I_\mu (1_{A_{j}})=\sum _{j=1}^m\xi _{j}\mu (A_{j})
\]
\end{proof}


\begin{thm}[Problem 9.5]
Let \((X,\mathcal{A},\mu )\) be a measure space and \(u\in \mathcal{M}^+(\mathcal{A})\). Show that the set-function
\[
A\mapsto \int 1_Aud\mu  \quad A\in \mathcal{A}
\]
is a measure.
\end{thm}

\begin{proof}
Set
\[
\nu :\mathcal{A}\rightarrow [0,\infty ]:A\mapsto \int 1_Aud\mu .
\]

\begin{enumerate}
  \item To show that \(\nu (\varnothing )=0\). Notice that \(1_\varnothing \equiv 0\).
  \item Let \(A=\bigcup _{j\in \Bbb{N}}A_{j}\) a disjoint union of sets \(A_{j}\in \mathcal{A}\). Note that
\[
\sum _{j=1}^\infty 1_{A_{j}}=1_A
\]
We have to show that
\begin{align*}
\nu (\bigcup _{j\in \Bbb{N}}A_{j})&=\int \bigg(\sum _{j=1}^\infty 1_{A_{j}}\bigg)\cdot ud\mu  \\
&=\int \bigg(\sum _{j=1}^\infty 1_{A_{j}}u\bigg)d\mu   \\
&= \sum _{j=1}^\infty \int 1_{A_{j}}ud\mu  \\
&= \sum _{j=1}^\infty  \nu (A_{j}).
\end{align*}
\end{enumerate}

\end{proof}



\begin{thm}[Problem 9.8]
Let \((X,\mathcal{A},\mu )\) be a measure space and \((u_{j})_{j\in \Bbb{N}}\subseteq \mathcal{M}^+(\mathcal{A}).\) If \(u_{j}\leq u\) for all \(j\in \Bbb{N}\) and some \(u\in \mathcal{M}^+(\mathcal{A})\) with \(\int ud\mu <\infty \), then
\[
\limsup_{j\in \infty }\int u_{j}d\mu \leq \int \limsup_{j\in \infty }u_{j}d\mu .
\]
\end{thm}

\begin{proof}
Showing that
\[
\limsup_{j\in \infty }\int u_{j}d\mu \leq \int \limsup_{j\in \infty }u_{j}d\mu 
\]
is equivalent with showing that
\[
-\liminf_{j\in \infty }\int -u_{j}d\mu \leq -\int \liminf_{j\in \infty }-u_{j}d\mu 
\]
which is equivalent with showing that
\[
\int \liminf_{j\in \infty }-u_{j}d\mu \leq \liminf_{j\in \infty }\int -u_{j}d\mu 
\]
which is equivalent with showing that
\[
\int ud\mu  + \int \liminf_{j\in \infty }-u_{j}d\mu \leq \int ud\mu  +\liminf_{j\in \infty }\int -u_{j}d\mu 
\]
which is equivalent with showing that
\[
\int u + \liminf_{j\in \infty }-u_{j}d\mu \leq \liminf_{j\in \infty }\bigg(\int ud\mu  +\int -u_{j}d\mu \bigg)
\]
which is equivalent with showing that
\[
\int  \liminf_{j\in \infty }\Big(u -u_{j}\Big)d\mu \leq \liminf_{j\in \infty }\bigg(\int u -u_{j}d\mu \bigg).
\]

By hypothesis, \(u_{j}\leq u\). So we have that \(u-u_{j}\) is a sequence of postive measurable functions and therefore our last statement follows by the theorem of Fatou.
\end{proof}
\newpage
\subsection{Measure Theory Chapter 7}

\begin{prop}
Let \((X,\mathcal{A}),(X',\mathcal{A}')\) be measurable spaces and let \(\mathcal{A}'=\sigma (\mathcal{G}').\) Then \(T:X\rightarrow X'\) is \(\mathcal{A}/\mathcal{A}'\)-measurable if and only if \(T^{-1}(\mathcal{G}')\subseteq \mathcal{A}.\)
\end{prop}

It suffices to assume \(T^{-1}(\mathcal{G}')\subseteq \mathcal{A}\) and show that
\[
T^{-1}(\mathcal{A})\subseteq \mathcal{A}.
\]
Consider \(\Sigma :=\{A'\subseteq X' : T^{-1}(A')\in \mathcal{A}\}\). We have that \(\mathcal{G}'\subseteq \Sigma \). To show that
\[
\mathcal{A}'\subseteq \Sigma ,
\]
it suffices to show that \(\Sigma \) is a \(\sigma \)-algebra.

\begin{enumerate}
  \item Showing that \(X'\in \Sigma \), is equivalent with showing that \(T^{-1}(X')\in \mathcal{A}\). Which holds as $T^{-1}(X')=X\in \mathcal{A}$.
  \item Showing that
\[
A'\in \Sigma  \Longrightarrow  A'^c\in \Sigma 
\]
is equivalent with showing that
\[
T^{-1}(A')\in \mathcal{A} \Longrightarrow  T^{-1}(A'^c)\in \mathcal{A} \qquad\checkmark
\]
  \item Showing that
\[
(A_{j}')_{j\in \Bbb{N}}\subseteq \Sigma  \Longrightarrow  \bigcup _{j\in \Bbb{N}}A_{j}\in \Sigma 
\]
is equivalent with showing that
\[
T^{-1}(A_{j}')\in \mathcal{A}\Longrightarrow T^{-1}\Big(\bigcup _{j\in \Bbb{N}}A_{j}\Big)\in \mathcal{A} \qquad\checkmark
\]
\end{enumerate}

\begin{prop}
Let \((X,\mathcal{A}),(X,\mathcal{A}')\) be measurable spaces and \(T:X\rightarrow X'\) be an \(\mathcal{A}/\mathcal{A}'\)-measurable map. For every measure \(\mu \) on \((X,\mathcal{A})\),
\[
\mu '(A'):=T(\mu )(A'):=\mu (T^{-1}(A')), \qquad A'\in \mathcal{A}'
\]
defines a measure on \((X',\mathcal{A}').\)
\end{prop}

\begin{proof}

\begin{enumerate}
  \item To show that
\[
\mu (T^{-1}(\varnothing ))=0 \qquad \checkmark
\]
  \item Assume \((A_{j}')_{j\in \Bbb{N}}\subseteq \mathcal{A}'\) mutually disjoint sets and show that
\begin{align*}
&\mu '\Big(\bigcup _{i=1}^nA_{i}'\Big)=\sum _{i=1}^n\mu '(A_{i}') \\
&\Uparrow  \text{[by definition of $\mu '$]}\\
&\mu (T^{-1}\Big(\bigcup _{i=1}^nA_{i}'\Big))=\sum _{i=1}^n\mu (T^{-1}(A_{i}'))\\
&\Uparrow [\text{$T^{-1}$ commutes with set operations}] \\
&\mu \Big(\bigcup _{i=1}^nT^{-1}(A_{i}')\Big)=\sum _{i=1}^n\mu (T^{-1}(A_{i}')) \\
&\Uparrow [\text{$\mu $ is measure on $(X,\mathcal{A})$ and $T$ is $\mathcal{A}/\mathcal{A}'$ measurable, given}] \\
&T^{-1}(A_{i}') \cap  T^{-1}(A_{j}')=\varnothing  \\
&\Uparrow [T^{-1} \text{ commutes with set operations}] \\
&A_{i}'\cap A_{j}'=\varnothing  \text{  [by hypothesis]}
\end{align*}
\end{enumerate}

\end{proof}
\begin{thm}[Problem 7.9i]
Let \(\mu \) be a measure on \((\Bbb{R},\mathcal{B})\). Show that
\[
F_\mu (x):=\begin{cases}\mu [0,x) &\text{ if }x>0 \\0 & \text{ if }x=0 \\ -\mu [x,0) &\text{ if }x<0\end{cases}
\]

\begin{enumerate}
  \item is monotonically increasing
  \item left-continuous function
\end{enumerate}
\end{thm}


\begin{proof}

\begin{enumerate}
  \item Showing that \(F_\mu \) is monotonically increasing is equivalent with showing that
\[
x\leq y \Longrightarrow  F_\mu (x)\leq F_\mu (y).
\]

\begin{enumerate}
  \item $x\leq 0\leq y:$. Then \(F_\mu (x)=-\mu [x,0)\leq 0\) and \(F_\mu (y)=\mu [0,y)\geq 0\).
  \item \(0<x\leq y:\) Then \([0,x)\subseteq [0,y)\). And \(\mu [0,x)\leq \mu [0,y)\).
  \item \(x\leq y<0:\) Then \([y,0)\subseteq [x,0).\) And \(\mu [y,0)\leq \mu [x,0)\).
\end{enumerate}
  \item Showing that \(F_\mu \) is left continuous is equivalent with assuming $(x_{k})$ a sequence such that \(x_{k}<x\) and \(x_{k}\uparrow x\) and showing that
\[
\lim_{k\rightarrow \infty }F_\mu (x_{k})=F_\mu (x).
\]
If if $x>0$, it suffices to show that
\[
\lim_{k\rightarrow \infty }\mu [0,x_{k})=\mu [0,x).
\]
If \(x<0\) it suffices to show that
\[
\lim_{k\rightarrow \infty }-\mu [x_{k},0)=-\mu [x,0).
\]
If \(x=0\) it suffices to show that
\[
\lim_{k\rightarrow \infty }-\mu [x_{k},0)=0.
\]
\end{enumerate}

Remember that:

\begin{enumerate}
  \item For any increasing sequence \((A_{j})_{j\in \Bbb{N}}\subseteq \mathcal{A}\) with \(A_{j}\uparrow A\in \mathcal{A}\) we have
\[
\mu (A)=\mu (\cup A_{j})=\lim_{j\in \infty }\mu (A_{j})
\]
  \item For any decreasing sequence \((A_{j})_{j\in \Bbb{N}}\subseteq \mathcal{A}\) with \(A_{j}\downarrow A\in \mathcal{A}\) we have
\[
\mu (A)=\mu (\cap A_{j})=\lim_{j\in \Bbb{N}}\mu (A_{j})
\]
\end{enumerate}
\end{proof}

\begin{thm}[Problem 7.9ii]
Let \(F:\Bbb{R}\rightarrow \Bbb{R}\) be a Stieltjes function. Show that
\[
\nu _F[a,b)=F(b)-F(a) \qquad \forall a,b\in \Bbb{R},a<b
\]
has a unique extension to a measure on \(\mathcal{B}\).
\end{thm}

\begin{proof}
By theorem 6.1 it suffices to show that \(\nu _F\) is a pre-measure. To show this it suffices to show that

\begin{enumerate}
  \item \(\nu _F(\varnothing )=v_F[a,a)=0\)
  \item \(\nu _F([a,b)\cup [b,c))=\nu _F([a,b))+\nu _F([b,c))\)
\begin{itemize}
  \item \begin{align*}
\nu _F([a,b))+\nu _F([b,c)) &= F(b)-F(a) + F(c)-F(b) \\
&=F(c)-F(a) \\
&=\nu _F[a,c) \\
&=\nu _F([a,b)\cup [b,c))
\end{align*}
\end{itemize}
  \item For any decreasing sequence \([a_{j},b_{j})_{j\in \Bbb{N}}\subseteq \mathcal{J}\) with \([a_{j},b)\downarrow [a,b)\in \mathcal{J}\) we have
\[
\nu _F([a,b))=\lim_{j\in \infty }\nu _F[a_{j},b).
\]
This last statement is equivalent with
\[
F(b)-F(a)=\lim_{j\in \infty }(F(b)-F(a_{j})).
\]
Note that since \([a_{j},b_{j})\downarrow [a,b)\in \mathcal{J}\) we have that \(a_{j}\uparrow a,a_{j}\leq a\) and therefore
\[
\lim_{j\in \infty }(F(b)-F(a_{j}))=F(b)-F(a),
\]
as \(F\) is left-continous.
  \item \(\mathcal{J}\) contains an exhausting sequence \([a_{j},b_{j})\) such that \([a_{j},b_{j})\uparrow \Bbb{R}\) and \(v_F[a_{j},b_{j})<\infty \)
\end{enumerate}

\end{proof}



